\documentclass[a4paper, 11pt]{report}

\usepackage[margin=1in]{geometry}

\usepackage[hidelinks]{hyperref}

\usepackage[utf8]{inputenc}
\usepackage[english,greek]{babel}
%\usepackage[LGR]{fontenc}

\usepackage{graphicx}
\graphicspath{ {images/} }

\usepackage[usenames, dvipsnames]{color}
\definecolor{mygray}{gray}{0.6}

%make pdflatex output copy-and-paste-able
\input{glyphtounicode}
\pdfgentounicode=1

%shortcuts
\newcommand{\gr}{\selectlanguage{greek}}
\newcommand{\en}{\selectlanguage{english}}

\usepackage[table,xcdraw]{xcolor}

\makeatletter
\renewenvironment{table}%
{\renewcommand{\familydefault}{\ttdefault}\selectfont
	\@float{table}}
{\end@float}
\makeatother

\usepackage{listings}
%\usepackage{pxfonts}

%\renewcommand{\ttdefault}{pcr}

\definecolor{verbgray}{gray}{0.93}

\definecolor{shadecolor}{rgb}{.9, .9, .9}

\lstnewenvironment{code}{%
	\lstset{%language=C,
		backgroundcolor=\color{verbgray},
		frame=single,
		framerule=0pt,
		basicstyle=\ttfamily,
		keywordstyle={{\color{ForestGreen}}},
		columns=fullflexible,
		morekeywords={prompt, \$}
		}
}{}




\begin{document}
\begin{center}{\small ΠΑΝΕΠΙΣΤΗΜΙΟ ΠΑΤΡΩΝ - ΤΜΗΜΑ ΜΗΧΑΝΙΚΩΝ Η/Υ ΚΑΙ ΠΛΗΡΟΦΟΡΙΚΗΣ }	
\end{center}

{\Large \noindent\textbf{Λειτουργικά Συστήματα}
\hfill \textbf{Άσκηση 3}}

\begin{flushright}
	\textbf{Εργασία Φοιτητή:}\\
	Δαμιανός Ντούμη - Σιγάλας, 6157\\
	\texttt{\en \href{mailto:nsigalas@ceid.upatras.gr}{nsigalas@ceid.upatras.gr}\gr}
\end{flushright}

\setlength{\parskip}{10pt}%

\vspace{-0.5cm}

\noindent \textbf{\en {\large Multi-process implementation} \gr}

\vspace{-0.4cm}
\noindent \en {\small \texttt{multiproc\_1.c} \hspace{0.5cm} \texttt{multiproc\_2.c}} \gr

Στα πρώτα δύο ερωτήματα της άσκησης καλούμαστε να υλοποιήσουμε τη ζητούμενη λειτουργικότητα χρησιμοποιώντας μια πολυδιεργασιακή προσέγγιση. Αρχικά αφού γίνει ο κατάλληλος έλεγχος ότι ο χρήστης έχει εισάγει τουλάχιστον ένα αλφαριθμητικό, τότε δημιουργούνται τόσες διεργασίες όσα και τα \en string \gr εισόδου  (\texttt{πλήθος = \en argc-2}) \gr κάθε μια από τις οποίες αναλαμβάνει να εκτυπώσει στην οθόνη \en rep \gr φορές \emph{μόνο} το \en string \gr που της αντιστοιχεί κάνοντας χρήση της \en \texttt{display(char*)} \gr που μας δίνεται. Σε αυτό το σημείο, χωρίς να έχει εφαρμοστεί κάποια τεχνική συγχρονισμού των διεργασιών, μια πιθανή έξοδος του προγράμματος είναι η επόμενη: 
%\vspace{-1cm}

\en \begin{code}
prompt $ ./multiproc_1 3 operating systems
[13914,139832146802432] o[13915,139832146802432] spyesrtaetmisn
[13915,139832146802432] sgys
[13914,139832146802432] otpeemrsa
[13915,139832146802432] styisntge
[13914,139832146802432] ompse prompt $
rating
\end{code}

\gr

Δύο προβλήματα γίνονται αμέσως αντιληπτά. Πρώτον, οι χαρακτήρες των \en string \gr εκτυπώνονται με ανακατεμένη σειρά και δεύτερον, το \en prompt \gr του \en bash \gr εκτυπώνεται σε λάθος σημείο. Το τελευταίο συμβαίνει γιατί η διεργασία-γονέας δεν περιμένει την ολοκλήρωση των παιδιών της, τερματίζοντας πριν από αυτά. Για την λύση του αρκεί να εξαναγκάσουμε τον γονέα να περιμένει όλες τις διεργασίες-παιδιά που έχει δημιουργήσει, με την χρήση της κλήσης συστήματος \en \texttt{wait} \gr.

Το πρώτο πρόβλημα δημιουργείται καθώς μεταξύ της εκτύπωσης κάθε χαρακτήρα η συνάρτηση \en \texttt{display} \gr προσθέτει τουλάχιστον 100 μικροδευτερόλεπτα αδράνειας στο \en thread \gr που την εκτελεί, με τον \en scheduler \gr του λειτουργικού συστήματος να αποφασίζει την αντικατάσταση της τρέχουσας διεργασίας με κάποια άλλη, η οποία θα συνεχίσει την λειτουργία της από το σημείο που έχει μείνει. Για την επίλυση του προβλήματος, πρέπει να επιτευχθεί ο αμοιβαίος αποκλεισμός (\en mutual exclusion\gr) των διεργασιών όταν εκτελούν τον κώδικα της συνάρτησης \en \texttt{display} \gr ώστε κάθε φορά μόνο μια διεργασία να γράφει ολοκληρωμένα το \en string \gr που της αντιστοιχεί στην οθόνη και έπειτα να επιτρέπει σε κάποια άλλη να πάρει τη θέση της. 

Για τον σκοπό αυτό, θα χρησιμοποιηθεί ένας σημαφόρος. Οι σημαφόροι αποτελούν μιά δομή δεδομένων που επιτρέπει των συγχρονισμό πολαπλών διεργασιών. 
Για την υλοποίηση του αμοιβαίου αποκλεισμού απαιτείται ένας δυαδικός σημαφόρος που παίρνει τιμές στο σύνολο \{0, 1\} και έχει αρχικοποιηθεί στην τιμή 1. Στην περίπτωση που κάποια χρονική στιγμή ο σημαφόρος έχει την τιμή 0, σημαίνει ότι κάποια άλλη διεργασία βρίσκεται στην κρίσιμη περιοχή και η διεργασία που προσπάθησε να μπει στην δική της κρίσιμη περιοχή πρέπει να περιμένει. Αντίθετα αν κάποια χρονική στιγμή ο σημαφόρος έχει την τιμή 1 σημαίνει οτι καμία διεργασία δεν βρίσκεται σε κρίσιμη περιοχή και η τρέχουσα διεργασία μπορεί να προχωρήσει στην προσπέλαση της.

\newpage

Για την υλοποίηση αυτής της λειτουργικότητας έχουν δημιουργηθεί οι εξής συναρτήσεις (οι οποίες χρησιμοποιούν το \en System V semaphore API\gr):
\vspace{-0.5cm}

\begin{enumerate}
\item \en \texttt{int create\_binary\_semaphore(key\_t key, int sem\_flags)} \gr

Δημιουργία του δυαδικού σημαφόρου με χρήση της \en \texttt{semget} \gr και αρχικοποίηση του στην \mbox{τιμή 1} με χρήση της \en \texttt{semctl}. \gr Επιστρέφεται το \en id \gr του δημιουργηθέντος σημαφόρου. 

\item \en \texttt{int destroy\_binary\_semaphore(int sem\_id)} \gr

Διαγραφή του σημαφόρου με χρήση της \en \texttt{semctl}, \gr για απελευθέρωση των πόρων του λειτουργικού συστήματος.

\item \en \texttt{int increment(int sem\_id)} \gr

Αύξηση της τιμής του σημαφόρου κατα 1 με χρήση της \en \texttt{semop} \gr ώστε κάποια από τις διεργασίες που βρίσκονται σε αναμονή (αν υπάρχουν) να ξεκινήσει την λειτουργία της.

\item \en \texttt{int decrement(int sem\_id)} \gr

Μείωση της τιμής του σημαφόρου κατά 1 με χρήση της \en \texttt{semop} \gr ώστε να μπλοκαριστεί η διεργασία σε περίπτωση που το αποτέλεσμα είναι αρνητικό (δηλαδή η τιμή του σημαφόρου \mbox{ήταν 0)}. 
\end{enumerate}

Με την χρήση αυτών των συναρτήσεων δημιουργείται ένας σημαφόρος αρχικοποιημένος στην τιμή 1. Πριν την έναρξη της κρίσιμης περιοχής του προγράμματός μας, δηλαδή την κλήση της \en \texttt{display} \gr ή και της \en \texttt{init} \gr στην περίπτωση του \en \texttt{multiproc\_2.c} \gr καλείται η \en \texttt{decrement} \gr ώστε αν η τιμή του σημαφόρου ήταν 1 να γίνει 0 και να επιτραπεί η εκτέλεση του κρίσιμου κώδικα \gr ενώ αν ήταν 0, να μπλοκαριστεί η διαδικασία μπαίνοντας σε αναμονή.

\noindent Πλέον μετά την εφαρμογή των παραπάνω τεχνικών μια πιθανή έξοδος του προγράμματος είναι η παρακάτω: 
%\vspace{-1cm}

\en \begin{code}
prompt $ ./multiproc_2 3 operating systems
STARTING: pid 28200, tid 140528943884032
STARTING: pid 28201, tid 140528943884032
[28200,140528943884032] operating
[28201,140528943884032] systems
[28200,140528943884032] operating
[28201,140528943884032] systems
[28200,140528943884032] operating
[28201,140528943884032] systems
prompt $
\end{code}


\vspace{0.7cm}
\noindent \textbf{\en {\large Multi-thread implementation \gr}}

\vspace{-0.4cm}
\noindent \en {\small \texttt{multithread\_1.c} \hspace{0.5cm} \texttt{multithread\_2.c}} \gr

Σε αυτά τα δύο προγράμματα αντί να δημιουργούνται τόσες διεργασίες όσα και τα \en string \gr που δίνει ο χρήστης ως ορίσματα κάθε μια εκ των οποίων αναλαμβάνει να εκτυπώσει \en rep \gr φορές το \en string \gr που της αντιστοιχεί, θα πρέπει να δημιουργείται ο αντίστοιχος αριθμός νημάτων (\en threads\gr) της ίδιας διεργασίας όπου το καθένα θα επιτελεί την ίδια εργασία.

Αφού κάθε νήμα πρέπει να εκτυπώνει \en rep \gr φορές το \en string \gr που του αντιστοιχεί για τον σκοπό αυτό δημιουργείται η συνάρτηση \texttt{\en void* thread\_runner(void *p)\gr} την οποία καλεί κάθε νήμα κατά την δημιουργία του. Παρακάτω παρατίθεται η έξοδος για μία εκτέλεση του προγράμματος \en \texttt{multithread\_2.c} \gr χωρίς να έχει εφαρμοσθεί κάποια τεχνική για τον συγχρονισμό των νημάτων.

\newpage

\en
\begin{code}
prompt $ ./multithread_2 3 operating systems
STARTING: pid 25839, tid 140057900754688
[25839,140057900754688] oSTARTING: pid 25839, tid 140057892361984
[25839,140057892361984] spyesrtaetmisn
[25839,140057892361984] sgy
[25839,140057900754688] ospteermast
[25839,140057892361984] siynsgt
[25839,140057900754688] oepmesr
ating
prompt $
\end{code}
\gr

Παρατηρούμε το αντίστοιχο πρόβλημα της εκτύπωσης χαρακτήρων από πάνω από ένα νήμα ταυτόχρονα. Αντίστοιχα με την λογική που αναπτύχθηκε παραπάνω, πρέπει να πετύχουμε τον αμοιβαίο αποκλεισμό των νημάτων έτσι ώστε μόνο ένα νήμα κάθε χρονική στιγμή να εκτελεί τις κρίσιμες περιοχές κωδικά του, δηλαδή στην περίπτωση μας τις συναρτήσεις \en \texttt{init} \gr και \en \texttt{display}. \gr 

Η βιβλιοθήκη \en \texttt{pthread} \gr μας παρέχει τα \en mutexes (\texttt{pthread\_mutex\_t}), \gr ένα εργαλείο με λειτουργία παρόμοια του δυαδικού σεμαφόρου. Το \en mutex \gr λειτουργεί σαν μια κλειδαριά με την διαφορά ότι μόνο ένα \en thread \gr κάθε χρονική στιγμή μπορεί να το κλειδώσει \en (lock) \gr και κανένα άλλο νήμα δεν μπορεί να το χρησιμοποιήσει μέχρι ο κάτοχος του να το ξεκλειδώσει \en (unlock). \gr Εξασφαλίζοντας ότι κατά την είσοδο του σε κρίσιμη περιοχή το νήμα δεσμεύει το \en mutex \gr επιτυγχάνουμε τον αμοιβαίο αποκλεισμό καθώς κανένα άλλο νήμα δεν μπορεί να το δεσμεύσει για την εκτέλεση της δικής του κρίσιμης περιοχής μέχρι το αρχικό να το αποδεσμεύσει.

Επιπλέον επειδή πρέπει να ολοκληρωθούν όλες οι κλήσεις της \en \texttt{init} \gr πριν ξεκινήσουν οι κλήσεις της \en \texttt{display} \gr από οποιοδήποτε νήμα, ανάμεσα στις δύο κρίσιμες περιοχές πρέπει να εισαχθεί ένα \en barrier (\en \texttt{pthread\_barrier\_t}\gr), το οποίο εξασφαλίζει ότι όλα τα \en threads \gr θα έχουν ολοκληρώσει μέχρι αυτό το σημείο, ώστε να συνεχίσουν έπειτα την λειτουργία τους. 

\noindent Μετά την εφαρμογή των παραπάνω, η έξοδος που παίρνουμε από μία εκτέλεση του \en \texttt{multithread\_2.c} \gr είναι η ακόλουθη:
\en
\begin{code}
prompt $ ./multithread_2 3 operating systems
STARTING: pid 25675, tid 140437636089600
STARTING: pid 25675, tid 140437627696896
[25675,140437627696896] systems
[25675,140437627696896] systems
[25675,140437627696896] systems
[25675,140437636089600] operating
[25675,140437636089600] operating
[25675,140437636089600] operating
prompt $
\end{code}
\gr

\end{document}

% Θεωρόντας ότι στο \en working directory \gr του χρήστη βρίσκεται το εκτελέσιμο του προγράμματος μας, αυτό θα πρέπει να εκτελεστεί με την ακόλουθη σύνταξη:
% 
% \vspace{-0.3cm}
% % Please add the following required packages to your document preamble:
% % \usepackage[table,xcdraw]{xcolor}
% % If you use beamer only pass "xcolor=table" option, i.e. \documentclass[xcolor=table]{beamer}
% \begin{table}[!h]
% 	\en
% 	\centering
% 	\label{my-label}
% 	\begin{tabular}{cccccc}
% 		./multiproc\_1                  & \#rep                                & string\_1                            & string\_2                            & ...                        & string\_n                               \\
% 		{\small \color[HTML]{009B55} argv{[}0{]}} & {\small \color[HTML]{009B55} argv{[}1{]}} & {\small \color[HTML]{009B55} argv{[}2{]}} & {\small \color[HTML]{009B55} argv{[}3{]}} & {\small \color[HTML]{009B55} ...} & {\small \color[HTML]{009B55} argv{[}argc-1{]}}
% 	\end{tabular}
% 	\gr
% \end{table}
% \vspace{-0.3cm}
% 
% \noindent δηλαδή πρώτο όρισμα %(μετά το όνομα του), 
% είναι ο αριθμός των φορών που κάθε διεργασία θα πρέπει να εκτυπώσει το \en string \gr που της αντιστοιχεί, ακολουθούμενος από άγνωστο αριθμό, αυθαίρετων \en string. \gr





