\documentclass[a4paper, 11pt]{report}

\usepackage[margin=1in]{geometry}

\usepackage[hidelinks]{hyperref}

\usepackage[utf8]{inputenc}
\usepackage[english,greek]{babel}
%\usepackage[LGR]{fontenc}

\usepackage{graphicx}
\graphicspath{ {images/} }

\usepackage[usenames, dvipsnames]{color}
\definecolor{mygray}{gray}{0.6}

%make pdflatex output copy-and-paste-able
\input{glyphtounicode}
\pdfgentounicode=1

%shortcuts
\newcommand{\gr}{\selectlanguage{greek}}
\newcommand{\en}{\selectlanguage{english}}

\usepackage[table,xcdraw]{xcolor}

\makeatletter
\renewenvironment{table}%
{\renewcommand{\familydefault}{\ttdefault}\selectfont
	\@float{table}}
{\end@float}
\makeatother

\begin{document}
\begin{center}{\small ΠΑΝΕΠΙΣΤΗΜΙΟ ΠΑΤΡΩΝ - ΤΜΗΜΑ ΜΗΧΑΝΙΚΩΝ Η/Υ ΚΑΙ ΠΛΗΡΟΦΟΡΙΚΗΣ }	
\end{center}

{\Large \noindent\textbf{Λειτουργικά Συστήματα}
\hfill \textbf{Άσκηση 3}}

\begin{flushright}
	\textbf{Εργασία Φοιτητή:}\\
	Δαμιανός Ντούμη - Σιγάλας, 6157\\
	\texttt{\en \href{mailto:nsigalas@ceid.upatras.gr}{nsigalas@ceid.upatras.gr}\gr}
\end{flushright}

\setlength{\parskip}{10pt}%

\vspace{-0.5cm}

\noindent \textbf{\en Multiproccess implementation \gr}

\vspace{-0.4cm}
\noindent \en {\small \texttt{multiproc\_1.c} \hspace{0.5cm} \texttt{multiproc\_2.c}} \gr

Στα πρώτα δύο ερωτήματα της άσκησης καλούμαστε να υλοποιήσουμε τη ζητούμενη λειτουργικότητα χρησιμοποιώντας μια πολυδιεργασιακή προσέγγιση. Θεωρόντας ότι στο \en working directory \gr του χρήστη βρίσκεται το εκτελέσιμο του προγράμματος μας, αυτό θα πρέπει να εκτελεστεί με την ακόλουθη σύνταξη:

\vspace{-0.3cm}
% Please add the following required packages to your document preamble:
% \usepackage[table,xcdraw]{xcolor}
% If you use beamer only pass "xcolor=table" option, i.e. \documentclass[xcolor=table]{beamer}
\begin{table}[!h]
	\en
	\centering
	\label{my-label}
	\begin{tabular}{cccccc}
		./multiproc\_1                  & \#rep                                & string\_1                            & string\_2                            & ...                        & string\_n                               \\
		{\small \color[HTML]{009B55} argv{[}0{]}} & {\small \color[HTML]{009B55} argv{[}1{]}} & {\small \color[HTML]{009B55} argv{[}2{]}} & {\small \color[HTML]{009B55} argv{[}3{]}} & {\small \color[HTML]{009B55} ...} & {\small \color[HTML]{009B55} argv{[}argc-1{]}}
	\end{tabular}
	\gr
\end{table}
\vspace{-0.3cm}

\noindent δηλαδή πρώτο όρισμα (μετά το όνομα του), είναι ο αριθμός των φορών που κάθε διεργασία θα πρέπει να εκτυπώσει το \en string \gr που της αντιστοιχεί, ακολουθούμενος από άγνωστο αριθμό, αυθαίρετων \en string. \gr

Αφού γίνει ο κατάλληλος έλεγχος ότι ο χρήστης έχει εισάγει τουλάχιστον ένα αλφαριθμητικό, τότε δημιουργούνται τόσες διεργασίες όσα και τα \en string \gr εισόδου  (\texttt{πλήθος = \en argc-2}) \gr κάθε μια από τις οποίες αναλαμβάνει να εκτυπώσει στην οθόνη \en rep \gr φορές \emph{μόνο} το \en string \gr που της αντιστοιχεί κάνοντας χρήση της \en \texttt{display(char*)} \gr που μας δίνεται. Σε αυτό το σημείο, χωρίς να έχει εφαρμοστεί κάποια τεχνική συχρονισμού των διεργασιών, μια πιθανή έξοδος του προγράμματος είναι η επόμενη: 




\end{document}