\documentclass[a4paper, 11pt]{report}

\usepackage[margin=1in]{geometry}

\usepackage[hidelinks]{hyperref}

\usepackage[utf8]{inputenc}
\usepackage[english,greek]{babel}
%\usepackage[LGR]{fontenc}

\usepackage{graphicx}
\graphicspath{ {images/} }

\usepackage[usenames, dvipsnames]{color}
\definecolor{mygray}{gray}{0.6}

%make pdflatex output copy-and-paste-able
\input{glyphtounicode}
\pdfgentounicode=1

%shortcuts
\newcommand{\gr}{\selectlanguage{greek}}
\newcommand{\en}{\selectlanguage{english}}

\usepackage[table,xcdraw]{xcolor}

\makeatletter
\renewenvironment{table}%
{\renewcommand{\familydefault}{\ttdefault}\selectfont
	\@float{table}}
{\end@float}
\makeatother

\usepackage{listings}
%\usepackage{pxfonts}

%\renewcommand{\ttdefault}{pcr}

\definecolor{verbgray}{gray}{0.93}

\definecolor{shadecolor}{rgb}{.9, .9, .9}

\lstnewenvironment{code}{%
	\lstset{language=C,
		backgroundcolor=\color{verbgray},
		frame=single,
		framerule=0pt,
		basicstyle=\ttfamily,
		keywordstyle={{\color{ForestGreen}}},
		columns=fullflexible,
		morekeywords={prompt, \$}
		}
}{}




\begin{document}
\begin{center}{\small ΠΑΝΕΠΙΣΤΗΜΙΟ ΠΑΤΡΩΝ - ΤΜΗΜΑ ΜΗΧΑΝΙΚΩΝ Η/Υ ΚΑΙ ΠΛΗΡΟΦΟΡΙΚΗΣ }	
\end{center}

{\Large \noindent\textbf{Λειτουργικά Συστήματα}
\hfill \textbf{Άσκηση 3}}

\begin{flushright}
	\textbf{Εργασία Φοιτητή:}\\
	Δαμιανός Ντούμη - Σιγάλας, 6157\\
	\texttt{\en \href{mailto:nsigalas@ceid.upatras.gr}{nsigalas@ceid.upatras.gr}\gr}
\end{flushright}

\setlength{\parskip}{10pt}%

\vspace{-0.5cm}

\noindent \textbf{\en Multiproccess implementation \gr}

\vspace{-0.4cm}
\noindent \en {\small \texttt{multiproc\_1.c} \hspace{0.5cm} \texttt{multiproc\_2.c}} \gr

Στα πρώτα δύο ερωτήματα της άσκησης καλούμαστε να υλοποιήσουμε τη ζητούμενη λειτουργικότητα χρησιμοποιώντας μια πολυδιεργασιακή προσέγγιση. Θεωρόντας ότι στο \en working directory \gr του χρήστη βρίσκεται το εκτελέσιμο του προγράμματος μας, αυτό θα πρέπει να εκτελεστεί με την ακόλουθη σύνταξη:

\vspace{-0.3cm}
% Please add the following required packages to your document preamble:
% \usepackage[table,xcdraw]{xcolor}
% If you use beamer only pass "xcolor=table" option, i.e. \documentclass[xcolor=table]{beamer}
\begin{table}[!h]
	\en
	\centering
	\label{my-label}
	\begin{tabular}{cccccc}
		./multiproc\_1                  & \#rep                                & string\_1                            & string\_2                            & ...                        & string\_n                               \\
		{\small \color[HTML]{009B55} argv{[}0{]}} & {\small \color[HTML]{009B55} argv{[}1{]}} & {\small \color[HTML]{009B55} argv{[}2{]}} & {\small \color[HTML]{009B55} argv{[}3{]}} & {\small \color[HTML]{009B55} ...} & {\small \color[HTML]{009B55} argv{[}argc-1{]}}
	\end{tabular}
	\gr
\end{table}
\vspace{-0.3cm}

\noindent δηλαδή πρώτο όρισμα %(μετά το όνομα του), 
είναι ο αριθμός των φορών που κάθε διεργασία θα πρέπει να εκτυπώσει το \en string \gr που της αντιστοιχεί, ακολουθούμενος από άγνωστο αριθμό, αυθαίρετων \en string. \gr

Αφού γίνει ο κατάλληλος έλεγχος ότι ο χρήστης έχει εισάγει τουλάχιστον ένα αλφαριθμητικό, τότε δημιουργούνται τόσες διεργασίες όσα και τα \en string \gr εισόδου  (\texttt{πλήθος = \en argc-2}) \gr κάθε μια από τις οποίες αναλαμβάνει να εκτυπώσει στην οθόνη \en rep \gr φορές \emph{μόνο} το \en string \gr που της αντιστοιχεί κάνοντας χρήση της \en \texttt{display(char*)} \gr που μας δίνεται. Σε αυτό το σημείο, χωρίς να έχει εφαρμοστεί κάποια τεχνική συχρονισμού των διεργασιών, μια πιθανή έξοδος του προγράμματος είναι η επόμενη: 
%\vspace{-1cm}

\en \begin{code}
prompt $ ./multiproc_1 3 operating systems
[13914,139832146802432] o[13915,139832146802432] spyesrtaetmisn
[13915,139832146802432] sgys
[13914,139832146802432] otpeemrsa
[13915,139832146802432] styisntge
[13914,139832146802432] ompse prompt $
rating
\end{code}

\gr

Σε αυτό το σημείο δύο προβλήματα γίνονται αμέσως αντιληπτά. Πρώτον, οι χαρακτήρες των \en string \gr εκτυπόνονται με ανακατεμένη σειρά και δεύτερον, το \en prompt \gr του \en bash \gr εκτυπώνεται σε λάθος σημείο. Το τελευταίο συμβαίνει γιατί η διεργασία-γονέας δεν περιμένει την ολοκλήρωση των παιδιών της, τερματίζοντας πριν από αυτά με αποτέλεσμα να εκτυπώνεται το \en prompt \gr πριν όλα τα παιδιά εκτυπώσουν όλους τους χαρακτήρες. Για την λύση του αρκεί να εξαναγκάσουμε τον γονέα να περιμένει πριν τον τερματισμό του, όλες τις διεργασίες-παιδιά που έχει δημιουργήσει, με την χρήση της κλήσης συστήματος \en \texttt{wait} \gr.

Το πρώτο πρόβλημα δημιουργείται καθώς μεταξύ της εκτύπωσης κάθε χαρακτήρα η συνάρτηση \en \texttt{display} \gr προσθέτει τουλάχιστον 100 μικροδευτερόλεπτα αδράνειας στο \en thread \gr που την εκτελεί, με τον \en scheduler \gr του λειτουργικού συστήματος να αποφασίζει να αντικαταστήσει την τρέχουσα διεργασία με κάποια άλλη, η οποία θα συνεχίσει την λειτουργία της από το σημείο που έχει μείνει πριν αντικατασταθεί (αν δεν πρόκειται για νέα διεργασία). Για την επίλυση του προβλήματος, πρέπει να επιτευχθεί ο αμοιβαίος αποκλεισμός (\en mutual exclusion\gr) των διεργασιών όταν εκτελούν τον κώδικα της συνάρτησης \en \texttt{display} \gr ώστε κάθε φορά μόνο μια διεργασία να γράφει ολοκληρωμένα το \en string \gr που της αντιστοιχεί στην οθόνη και έπειτα να επιτρέπει σε κάποια άλλη να πάρει τη θέση της. 

Για τον σκοπό αυτό, θα χρησιμοποιηθεί ένας σημαφόρος. Οι σημαφόροι αποτελούν μιά δομή δεδομένων που επιτρέπει των συγχρονισμό πολαπλών διεργασιών. Για την υλοποίηση του αμοιβαίου αποκλεισμού απαιτείται ένας δυαδικός σημαφόρος που παίρνει τιμές στο σύνολο \{0, 1\} και έχει αρχικοποιηθεί στην τιμή 1. Στην περίπτωση που κάποια χρονική στιγμή ο σημαφόρος έχει την τιμή 0, σημαίνει ότι κάποια άλλη διεργασία βρίσκεται στην κρίσημη περιοχή και η διεργασία που προσπάθησε να μπεί στην δική της κρίσιμη περιοχή πρέπει να περιμένει. Αντίθετα αν κάποια χρονική στιγμή ο σημαφόρος έχει την τιμή 1 σημαίνει οτι καμία διεργασία δεν βρίσκεται σε κρίσημη περιοχή και η τρέχουσα διεργασία μπορεί να προχωρήσει στην προσπέλαση της.

Για την υλοποίηση αυτής της λειτουργικότητας έχουν δημιουργηθεί οι εξής συναρτήσεις (οι οποίες χρησιμοποιούν το \en System V semaphore API\gr):
\vspace{-0.5cm}

\begin{enumerate}
\item \en \texttt{int create\_binary\_semaphore(key\_t key, int sem\_flags)} \gr

Δημιουργία του δυαδικού σημαφόρου με χρήση της \en \texttt{semget} \gr και αρχικοποιησή του στην \mbox{τιμή 1} με χρήση της \en \texttt{semctl}. \gr Επιστρέφεται το \en id \gr του δημιουργηθέντος σημαφόρου. 

\item \en \texttt{int destroy\_binary\_semaphore(int sem\_id)} \gr

Διαγραφή του σημαφόρου με χρήση της \en \texttt{semctl}, \gr για απελευθέρωση των πόρων του λειτουργικού συστήματος.

\item \en \texttt{int increment(int sem\_id)} \gr

Αύξηση της τιμής του σημαφόρου κατα 1 με χρήση της \en \texttt{semop} \gr ώστε κάποια από τις διεργασίες που βρίσκονται σε αναμονή (αν υπάρχουν) να ξεκινήσει την λειτουργία της.

\item \en \texttt{int decrement(int sem\_id)} \gr

Μείωση της τιμής του σημαφόρου κατα 1 με χρήση της \en \texttt{semop} \gr ώστε να μπλοκαριστεί η διεργασία σε περίπτωση που το αποτέλεσμα είναι αρνητικό (δηλαδή η τιμή του σημαφόρου \mbox{ήταν 0)}. 
\end{enumerate}

Με την χρήση αυτών των συναρτήσεων δημιουργείται ένας σημαφόρος αρχικοποιημένος στην τιμή 1. Πριν την έναρξη της κρίσημης περιοχής του προγράμματός μας, δηλαδή την κλήση της \en \texttt{display} \gr ή και της \en \texttt{init} \gr στην περίπτωση του \en \texttt{multiproc\_2.c} \gr καλείται η \en \texttt{decrement} \gr ώστε αν η τιμή του σημαφόρου ήταν 1 να γίνει 0 και να επιτραπεί η εκτέλεση του κρίσιμου κώδικα \gr ενώ αν ήταν 0, να μπλοκαριστεί η διαδικασία μπαίνοντας σε αναμονή.

Πλέον μετά την εφαρμογή των παραπάνω τεχνικών μια πιθανή έξοδος του προγράμματος είναι η παρακάτω: 
%\vspace{-1cm}

\en \begin{code}
prompt $ ./multiproc_1 3 operating systems
[24049,140003813525248] operating
[24050,140003813525248] systems
[24049,140003813525248] operating
[24050,140003813525248] systems
[24049,140003813525248] operating
[24050,140003813525248] systems
prompt $
\end{code}


\end{document}